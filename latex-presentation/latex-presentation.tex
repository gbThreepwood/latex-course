\documentclass[10pt,a4paper]{beamer}
%\usepackage[utf8]{inputenc}
\usepackage[T1]{fontenc}
\usepackage[nynorsk]{babel}
\usepackage{amsmath}
\usepackage{amssymb}
\usepackage{graphicx}


\usepackage[
backend=biber,
style=ieee,
sorting=ynt
]{biblatex}

\addbibresource{bib/latex-kurs-referansar.bib}

\usepackage{svg}

\usepackage{subfiles}

\usepackage{multicol}

\usetheme{metropolis}

\setbeamertemplate{section in toc}[sections numbered]
\setbeamertemplate{subsection in toc}[subsections numbered]

\usepackage[pagewise]{lineno}
\usepackage{csquotes}

\usepackage{minted}
\usepackage{listings}

\usepackage{multirow}

\usepackage{siunitx}
\usepackage{steinmetz}

\usepackage{xcolor}

\usepackage{enumerate}

\usepackage{lipsum}

\usepackage{caption}
\usepackage{subcaption}

\usepackage{pgfplots}


\title{Grunnleggande innføring i \LaTeX}
\date{\today}
\author{Eirik Haustveit}
\institute{Institutt for datateknologi, elektroteknologi og realfag}
\begin{document}
	
	\titlepage

	\section{Plan for kurset}

	\begin{frame}{Plan}
		
          \begin{itemize}
                        \item Motivasjon (kvifor bruka \LaTeX{})
			\item Kort om bakrunnen for utviklinga av \LaTeX{}
			\item Grunnleggande gjennomgang av verkemåte
			\item Gjennomgang av nokon viktige grunnleggande funksjonar
			\item Typesetting av matematikk
			\item Deling av dokumentet i fleire filar
			\item Referanselister
			\item Demonstrasjon av nokre avansert teknikkar (vist me får tid)
		\end{itemize}
		
	\end{frame}


	\subfile{tex/motivation.tex}
	
	\subfile{tex/history.tex}
	
	\subfile{tex/introduction.tex}

	\subfile{tex/basic-usage.tex}
	
	\subfile{tex/basic-commands.tex}
	
	\subfile{tex/mathematics.tex}
	
	\subfile{tex/multi-file-documents.tex}

	\subfile{tex/figures-and-tables.tex}

	\subfile{tex/references.tex}


        \subfile{tex/advanced-tools.tex}


	\begin{frame}{Vidare lesing}
	Nicola Tesla har laga ein fin artikkel \cite{tesla1888}.
	
	Denne presentasjonen har forsøkt å gi ein introduksjon til grunnleggande \LaTeX{} for at ein skal komma i gang med skrivinga. Det er mange avanserte tema som ikkje er tatt med.
	
	\begin{itemize}
		\item Bruk av teljarar
		\item Generering av ordlister og symbollister
		\item Plotting av grafar med pgfplots
		\item Teikning av elektriske krinsar med circuitikz
		\item Generell tekning med TikZ
		\item Programmering i \TeX{}
		\item Utvikling av eigne malar og andre pakkar
		\item Generering av \LaTeX{} frå andre program (Python, Matlab, etc.)
	\end{itemize}
	
	
	\end{frame}

	\begin{frame}{Referansar}	
	
	\printbibliography 
	
	\end{frame}
	
	
\end{document}
%%% Local Variables:
%%% mode: latex
%%% TeX-master: t
%%% End:

\section{Motivasjon}

\begin{frame}{Motivasjon}
Når ein skal i gang med skriving av ein større tekst som til dømes ein bachelorrapport så vil nok mange ukritisk velga Word (eller meir spesifikt Microsoft Word). Eg trur grunnen til dette først og fremst er at ein velger noko ein kan, eller noko ein allereie har høyrt om. Det finnes andre alternativ:

\begin{itemize}
	\item ONLYOFFICE
	\item ConTeXT
	\item Org-mode
	\item Pandoc
	\item \LaTeX
        \item AsciiDoc
        \item typst
\end{itemize}

Og mange mange fleire...

\end{frame}


\begin{frame}{Motivasjon}

Denne presentasjonen omhandlar \LaTeX, som kanskje er det mest populære alternativet til Word for større tekstar. Det er ganske stor forskjell i korleis ein jobber i \LaTeX, samanlikna med Word. Begge tilnærminane har sine eigne fordelar og ulemper. Eg opplever at fordelane med \LaTeX er større enn ulempane, men det viktigaste er at ein gjer eigne vurderingar, og ikkje berre fylgjer straumen.
	
\end{frame}

%%% Local Variables:
%%% mode: latex
%%% TeX-master: "../latex-presentation"
%%% End:

\section{Figurar og tabellar}

	\begin{frame}{Figurar (bilete)}
	
	Høgskulen sin logo er gitt i figur \ref{fig:hvl-logo}.
	
	\begin{figure}
		\includegraphics[scale=0.1]{img/HVL-logo-vert-pos-norsk.png}
		\caption{HVL sin logo}
		\label{fig:hvl-logo}
	\end{figure}
	
\end{frame}


\begin{frame}[containsverbatim]{Figurar (bilete)}
	
	\begin{minted}[breaklines]{latex}
Høgskulen sin logo er gitt i figur \ref{fig:hvl-logo}.

\begin{figure}
	\includegraphics[scale=0.1]{img/HVL-logo-vert-pos-norsk.png}
	\caption{HVL sin logo}
	\label{fig:hvl-logo}
\end{figure}
	\end{minted}
	
\end{frame}

\begin{frame}[containsverbatim]{Underfigurar}
	
	Det kan vera nyttig å gruppera ein figur i fleire underfigurar. Til dømes eit krinsskjema og ein graf som beskriv responsen til krinsen.
	
	\textbf{Ikkje bruk denne utdaterte pakken:} \mintinline{latex}|\usepackage{subfig}|. Den var sist oppdatert i 2005.
	
	I staden bør du bruka:
	
	\begin{minted}{latex}
\usepackage{caption}
\usepackage{subcaption}
	\end{minted}
	
	
\end{frame}


\begin{frame}[containsverbatim]{Underfigurar}
	
	\begin{figure}
		\centering
		\begin{subfigure}[b]{0.3\textwidth}
			\centering
			\includegraphics[width=\textwidth]{example-image-a}
			\caption{Den første figuren}
			\label{fig:first-fig}
		\end{subfigure}
		\hfill
		\begin{subfigure}[b]{0.3\textwidth}
			\centering
			\includegraphics[width=\textwidth]{example-image-b}
			\caption{Den andre figuren}
			\label{fig:second-fig}
		\end{subfigure}
		\hfill
		\begin{subfigure}[b]{0.3\textwidth}
			\centering
			\includegraphics[width=\textwidth]{example-image-c}
			\caption{Den tredje figuren}
			\label{fig:third-fig}
		\end{subfigure}
		\caption{Tre figurar}
		\label{fig:three-figures}
	\end{figure}
	
	Figur \ref{fig:first-fig} syner ein A, medan figur \ref{fig:second-fig} syner ein B.
	
\end{frame}


\begin{frame}[containsverbatim]{Underfigurar}
	
	\begin{minted}{latex}
\begin{figure}
	\centering
	\begin{subfigure}[b]{0.3\textwidth}
		\centering
		\includegraphics....
		\caption{Den første figuren}
		\label{fig:first-fig}
	\end{subfigure}
	\hfill
	\begin{subfigure}[b]{0.3\textwidth}
		\centering
		\includegraphics....
		\caption{Den andre figuren}
		\label{fig:second-fig}
	\end{subfigure}
	...
\end{figure}
	\end{minted}
	
	
	
	
\end{frame}


\begin{frame}[containsverbatim]{Tabellar}
	
	Tabellar kan opprettast i ``tabular'' miljøet.
	
	\begin{table}
		\caption{Teknisk tabell}
		\begin{tabular}{ |c|c|c| } 
			\hline
			kolonne1 & kolonne2 & kolonne3 \\
			\hline
			\multirow{3}{4em}{Fleire rader} & celle2 & celle3 \\ 
			& celle5 & celle6 \\ 
			& celle8 & celle9 \\ 
			\hline
		\end{tabular}
	\end{table}
	
	\begin{minted}{latex}
\begin{tabular}{ |c|c|c| } 
	\hline
	kolonne1 & kolonne2 & kolonne3 \\
	\hline
	\multirow{3}{4em}{Fleire rader} & celle2 & celle3 \\ 
	& celle5 & celle6 \\ 
	& celle8 & celle9 \\ 
	\hline
\end{tabular}
	\end{minted}
	
\end{frame}

%\begin{frame}[containsverbatim]{Tabellar}
%	
%	For å laga litt finare tabellar kan ein nytta: \mintinline{latex}|\usepackage{booktabs}|.
%	
%\end{frame}
	\begin{frame}{Innhaldsfortegnelse}
	
	Ei innhaldsfortegnelse i \LaTeX{} vert normalt autogenerert ut frå inndelinga i seksjonar og kapittel. Kommandoen \mintinline{latex}{\tableofcontents} plasserer innhaldsfortegnelsen der ein måtte ynskja, til dømes her:
	
	\tableofcontents
\end{frame}

\begin{frame}{Fotnoter}
	
	Ei fotnote er eit notat til ein tekst plassert nederst på sida, med eit teikn i brødteksten som refererer til fotnoten\footnote{Fotnoter er ofte nytta som eit alternativ til å inkludera ei lang forklaring inne i brødteksten. Ein lesar som forstår teksten kan då hoppa over fotnoten, medan ein som har problem med å forstå teksten kan få hjelp.}. Kommandoen for å setta inn ei fotnote må plasserast der ein ynskjer å plassera referansen:
	
	\mintinline{latex}{..refererer til fotnoten\footnote{Fotnoter er ofte...}.}
	
\end{frame}

\begin{frame}{Kryssreferansar}
	
	Kryssreferering er interne referansar i eit dokument for å fortelja lesaren kvar han kan finna gitt informasjon i dokumentet. Til dømes:
	
	\begin{displayquote}
		Ohms lov er gitt i formel \eqref{eq:ohms-law}. Bla bla bla, meir tekst.... meir tekst meir tekst
		
		\begin{equation}
			U = R \cdot I
			\label{eq:ohms-law}
		\end{equation}
		
	\end{displayquote}
	
	Eit anna relevant døme er:
	
	\begin{displayquote}
		Grunnleggande om historien til \TeX og \LaTeX er gitt i avsnitt \ref{sec:history}.
	\end{displayquote}
	
\end{frame}



\begin{frame}[containsverbatim]{Skrifttypar og skriftstorleik}
	
	Val av skrifttype kan ha stor betydning for korleis ein tekst vert oppfatta. Standardisering av skrifttypar vart nødvendig etter at Johann Gutenberg oppfant boktrykkerkunsten. Den første standardiserte skrifttypen til Gutenberg var Textura.
	
	\tgothfamily Dette er tekst i Textura font.\normalfont
	
	Ein kan kategorisera skrifttypar i ulike kategoriar. Nokon vanlege kategoriar er:
	
	\begin{itemize}
		\item \textit{Kursiv}
		\item \textbf{Bold}
		\item \textrm{Serif (bokstavane har føtter)}
		\item \textsf{Sans-serif (ikkje-serif)}
		\item \texttt{Monospace (kvar bokstav tek like mykje plass)}
	\end{itemize}
	
%{\fontfamily{qcr}\selectfont
%Dette er TeX Gyre Cursor
%}

	
\end{frame}


\begin{frame}[containsverbatim]{Skrifttypar og skriftstorleik}
	
	Ein del skrifttypar er kategorisert som ``Metric-compatible fonts'', som betyr at dei har samme storleik som andre skrifttypar i samme kategori. Typisk er det åpne (gratis) skrifttypar som kan erstatta kommersielle typar.
		
	Det er nokon konkrete skrifttypar som er meir utbredt enn andre, og dei mest kjente har som regel fleire alternativ:
	
	\begin{itemize}
		\item Helvetica (også kjent som Arial eller Liberation sans)
		\item Times (også kjent som Times New Roman)
		\item Palatino (Antiqua).
		\item Computer Modern
	\end{itemize}
	

	
\end{frame}


\begin{frame}[containsverbatim]{Skrifttypar og skriftstorleik}
	
\LaTeX{} nyttar skrifttype-familien Computer Modern som standard. Familien har ulike konkrete skrifttypar tilgjengeleg:
	
\textrm{Døme på tekst med roman (vanleg) font}\\
\textbf{Døme på tekst med bold font}\\
\textit{Døme på tekst med kursiv (italic) font}\\
\textsf{Døme på tekst med sans serif font}\\
\texttt{Døme på tekst med typewriter (monospace) font}

	\begin{minted}{latex}
\textrm{Døme på tekst med roman (vanleg) font}
\textsf{Døme på tekst med sans serif font}
\texttt{Døme på tekst med typewriter (monospace) font}
	\end{minted}
	
	For å endra skrifttype (font) i heile dokumentet kan ein importera ulike pakkar, til dømes: \mintinline{latex}|\usepackage{tgbonum}|
	
	\url{https://tug.org/FontCatalogue/}
	
	
\end{frame}


\begin{frame}[containsverbatim]{Skriftstorleik}
	
Ein bør normalt overlata til \LaTeX{} å bestemma kva som er passande skriftstorleik for overskrifter, fotnoter, brødtekst, m.m. Dersom ein likevel ynskjer å overstyra dette kan ein nytta {\huge stor} og {\tiny liten} skrift i samme tekst.
	
	%TODO: Trekk inn noko av det som står her: https://www.overleaf.com/learn/latex/Font\_sizes
	
\end{frame}

\begin{frame}{Skriving på norsk}
	
	For å skiva \LaTeX{} på norsk er det nokon pakkar ein med fordel kan importera.
	
	\mintinline{latex}|\usepackage[nynorsk]{babel}| velger at autogenerert tekst som til dømes tittelen ``Innhald'' i innhaldsfortegnelsen vert på nynorsk.
	
	\mintinline{latex}|\usepackage{parskip}| sørger for at avsnitt vert delt med linjeskift, som er vanleg i norge.
	
\end{frame}

\begin{frame}[containsverbatim]{Kolonner}
	
	Dokumentet kan delast opp i kolonner ved hjelp av \mintinline{latex}|\usepackage{multicol}|
	
	\begin{columns}
		\begin{column}{0.49\textwidth}
			
			\begin{minted}{latex}
\begin{multicols}{2}[]
	the quick brown...
\end{multicols}
			\end{minted}
			
		\end{column}
		% Column 2 (vertical line)
		\begin{column}{.02\textwidth}
			\rule{.1mm}{0.7\textheight}
		\end{column}
		% Column 3    
		\begin{column}{0.245\textwidth}
			The quick brown fox jumps over the lazy dog the quick brown fox jumps over the lazy dog	the quick brown fox jumps over the lazy dog
		\end{column}
		\begin{column}{0.245\textwidth}
			The quick brown fox jumps over the lazy dog	the quick brown fox jumps over the lazy dog the quick brown fox jumps over the lazy dog
		\end{column}
	\end{columns}
	
	
	%		\begin{multicols}{3}
		%			[
		%			\section{First Section}
		%			All human things are subject to decay. And when fate summons, Monarchs must obey.
		%			]
		%			test
		%			fasfd
		%		\end{multicols}
	
\end{frame}

\begin{frame}[containsverbatim]{Fargar}
	
	
	\color{red} Tekst \color{green} kan \color{blue} skrivast \color{purple} i \color{yellow} ulike \color{orange} fargar.
	
	{\color{red} \rule{\linewidth}{0.5mm}}
	
	\color{black}
	
	For å få til dette kan ein nytta \mintinline{latex}|\usepackage{xcolor}|
	
	\begin{minted}[breaklines]{latex}
\color{red} Tekst \color{green} kan \color{blue} skrivast \color{purple} i \color{yellow} ulike \color{orange} fargar.
	\end{minted}
	
\end{frame}

\begin{frame}[containsverbatim]{Punktlister og nummererte lister}
	
Punktlister på denne formen:

	\begin{itemize}
		\item Eit element
		\item Eit element til
	\end{itemize}

får ein med:
	
	\begin{minted}{latex}
\begin{itemize}
	\item Eit element
	\item Eit element til
\end{itemize}
	\end{minted}

Nummererte lister får ein med \mintinline{latex}|\begin{enumerate}|:

	\begin{enumerate}
			\item Eit element
			\item Eit element til
	\end{enumerate}
	
\end{frame}


%\begin{frame}[containsverbatim]{Punktlister og nummererte lister}
%	
%	\begin{enumerate}
%		\item Eit element
%		\item Eit element til
%	\end{enumerate}
%	
%	\begin{minted}{latex}
%\begin{enumerate}
%	\item Eit element
%	\item Eit element til
%\end{enumerate}
%	\end{minted}
%	
%	\begin{enumerate}
%		\item[*] Eit element
%		\item[!] Eit element til
%		\item Enda eit
%	\end{enumerate}
	
%	\begin{minted}{latex}
%\begin{enumerate}
%	\item[*] Eit element
%	\item[!] Eit element til
%	\item Enda eit
%\end{enumerate}
%	\end{minted}
%	
%\end{frame}



	\begin{frame}[containsverbatim]{hyperref pakken}
	
	Pakken ``hyperref'' kan nyttast for å laga linkar i innhaldsfortegnelse, kryssrefferansar, m.m.:
	
	\begin{minted}{latex}
\usepackage{hyperref}
\hypersetup{
	colorlinks=true,
	linkcolor=blue,
	filecolor=magenta,      
	urlcolor=cyan,
	pdftitle={Døme på bruk av hyperref},
	pdfpagemode=FullScreen,
}
	\end{minted}

\end{frame}





\begin{frame}[containsverbatim]{Måleeiningar}
	
	Pakken siunitx (\mintinline{latex}|\usepackage{siunitx}|) kan nyttast for å forenkla arbeidet med konsekvent typesetting av SI einingar.
	
	\mintinline{latex}|\(U = \SI{230}{\kilo\volt}\)| gir: \(U = \SI{230}{\kilo\volt}\)
	
	\mintinline{latex}|\(\SI{15}{\kilo\meter\cubed\per\second}\)| gir: \(\SI{15}{\kilo\meter\cubed\per\second}\)
	
	Fasevektorar kan typesettast ved hjelp av:
	\begin{minted}[breaklines]{latex}
\(\SI[parse-numbers=false]{38\phase{34^{\circ}}}{\ampere}\)
	\end{minted}
	\mintinline{latex}||
	Som gir: \( \SI[parse-numbers=false]{38\phase{34^{\circ}}}{\ampere}\)

        Pakken siunitx er ein avansert pakke som støttar mange ulike standardar for typesetting av einingar (brukarmanualen er på 82 sider).
        
\end{frame}




\begin{frame}[containsverbatim]{Listing av kjeldekode}
	
	Kjeldekode i ulike programmeringsspråk vert normalt lista i ulike fargar for å betra lesbarheten.
	
	Til dømes i Matlab:
	
	\begin{minted}{matlab}
% Generate a random number
a = randi(100, 1);

% If it is even, divide by 2
if rem(a, 2) == 0
disp('a is even')
b = a/2;
end
	\end{minted}
	
\end{frame}




\begin{frame}[containsverbatim]{Listing av kjeldekode}
	Ein har ulike pakkar for listing av kjeldekode.
	
	\begin{itemize}
		\item verbatim
		\item listings
		\item Minted
	\end{itemize}
	
	Den mest avanserte (beste?) er Minted, men den krever at ein har installert Python. Dersom ein ikkje har avanserte behov kan ein nytta listings.
	
	\begin{lstlisting}
\begin{minted}{latex}
	
\end{minted}
	\end{lstlisting}
	
\end{frame}


\begin{frame}[containsverbatim]{Importering av eksterne filar}
	
	Importering av eksten kjeldekode gjer det enklare å holda dokumentet oppdatert. Endringar i den kjeldekoden du jobbar med vil umiddelbart bli reflektert i rapporten som skal dokumentera koden:
	
	\begin{minted}{latex}
\inputminted[<options>]{<language>}{<filename>}
	\end{minted}
	
	Til dømes vil \mintinline{latex}|\inputminted{csharp}{source-code/hello.cs}| i denne presentasjonen gi oss:
	
	\inputminted{csharp}{source-code/hello.cs}
	
\end{frame}
%%% Local Variables:
%%% mode: latex
%%% TeX-master: "../latex-presentation"
%%% End:


\section{Organisering i fleire filar}

	\begin{frame}{Organisering av dokumentet i fleire filar}
	
	For å betra organiseringa av eit dokument, og for å forenkla samarbeid kan dokumentet delast opp i fleire filar. Til dømes ein fil for kvart kapittel. \LaTeX støtter ulike kommandoar for innhenting av eksterne filar\footnote{Merk at det her er snakk om filar med \LaTeX-kode, ikkje kjeldekode som skal listast opp.}
	
	\begin{itemize}
		\item \mintinline{latex}|\input{vedlegg.tex}|
		\item \mintinline{latex}|\include{vedlegg.tex}|
		\item \mintinline{latex}|\import{vedlegg.tex}|
		\item \mintinline{latex}|\subfile{vedlegg.tex}|
	\end{itemize}
	
\end{frame}

\begin{frame}{Organisering av dokumentet i fleire filar}
	
	Forskjellen på \mintinline{latex}|\input{vedlegg.tex}|, og \mintinline{latex}|\include{vedlegg.tex}| er at den siste setter inn eit sideskift før og etter innhenting av den eksterne filen (kommandoen for å gjera dette sideskiftet manuelt er: \mintinline{latex}|\clearpage|).
	
	Kommandoen \mintinline{latex}|\import{vedlegg.tex}| er ikkje innebygd, og krever at ein legg til \mintinline{latex}|\usepackage{import}| i preamble.
	
	Det samme gjeld \mintinline{latex}|\subfile{vedlegg.tex}|, som krev \mintinline{latex}|\usepackage{subfiles}|
	
	%Import gjer det enklare å organisera større \LaTeX{} prosjekt. \mintinline{latex}|\subimport{}|
	
\end{frame}

\begin{frame}{Organisering av dokumentet med subfile}

``import'' pakken gjer det enklare å organisera dokumentet ved at kvart kapittel kan ha si eige mappe med både ``*.tex'' fil, og eksterne filar for figurar m.m.
  
	Ved hjelp av ``subfiles'' pakken kan du bygga kvar ekstern fil individuelt, og dei vil automatisk nytta samme ``preamble'' som hovudfilen. Dette er ein veldig fleksibel pakke for å dela eit større prosjekt inn i mange filar.
	
\end{frame}
%%% Local Variables:
%%% mode: latex
%%% TeX-master: "../latex-presentation"
%%% End:

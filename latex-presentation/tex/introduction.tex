\section{Introduksjon til \LaTeX}

\begin{frame}{Bruksområde for \LaTeX}
	
	\LaTeX{} er eit programvaresystem for å laga ulike typar tekstdokument. Det er på mange måtar eit alternativ til Microsoft Word men med ein svært ulik tilnærming til brukargrensesnitt.
	
	Det kan mellom anna nyttast for:
	
	\begin{itemize}
		\item Artiklar, rapportar, bøker m.m.
		\item Presentasjonar
		\item Typesettig av matemtaikk for import i andre program
		\item Teikning av figurar og grafar
	\end{itemize}
	
\end{frame}


	\begin{frame}{\LaTeX{} vs Microsoft Word}
	
	Microsoft Word (og mange andre liknande program som til dømes Libre Office) er såkalla ``what you see is what you get'' (WYSIWYG) tekstredigeringsprogram. Det du ser ved redigering er (omtrent) det samme som det endelige resultatet. %Det meste av funksjonaliteten er tilgjengeleg i menyar, men det betyr ikkje nødvendigvis at det er enkelt å finna fram til ein gitt funksjon.
	
	\LaTeX{} er eit ``what you see is what you mean'' (WYSIWYM) tekstredigeringssystem. Ein markerer teksten med spesielle kommandoar for å angi meininga, og så er det opp til \LaTeX{} å tolke denne meininga og presentera resultatet. Til dømes får ein \textbf{utheva tekst}, ved hjelp av markeringa \mintinline{latex}|\textbf{utheva tekst}|
	
	Det er truleg enklare å komma i gang med Word enn \LaTeX{}, og dette er kanskje ein viktig grunn til at Word er meir utbredt\footnote{Lobbyisme frå Mircrosoft er derimot truleg ein mykje viktigare grunn.}. Det er likevel mange fordelar med \LaTeX{} som me kjem tilbake til seinare i presentasjonen.
	
\end{frame}

\begin{frame}{Skyløysingar}

  	\includesvg[width=1.5in]{img/overleaf-logo.svg}
	
  Det er fleire ulike skyløysingar tilgjengelig for å skriva i \LaTeX. Dette har fordelen at det er raskt å komma i gang utan at du må bruka tid på å installera og konfigurera programmvare lokalt.

  Ulempen med slike løysingar er først og fremst at ein legg alle filane sine på \textit{nokon andre si datamaskin}. Ein anna ulempe er at det kan vera litt treigare og at det ikkje alltid er oppdatert til siste versjon av \LaTeX. Det er også eit problem at ein ikkje kan jobba på dokumentet utan å ha tilgong til internett.
	
	Ei populær skyløysing er Overleaf: \url{https://www.overleaf.com/}. Dersom du ikkje har installert programvaren før oppmøte på kurset i dag så kan du kjapt oppretta ein brukar her for å testa ut det som vert demonstrert.
	
\end{frame}


\begin{frame}{Installasjon av programvare}

	\includesvg[width=1.5in]{img/miktex-logo.svg}
  
	Framgangsmåten for lokal installasjon av \LaTeX{} vil variera avhengig av kva operativsystem ein nyttar.
	
	For Windows kan ein installera MikTeX: \url{https://miktex.org/download}.
	
	Ein treng også ein teksteditor, og her er TeXstudio eit godt alternativ: \url{https://www.texstudio.org/}
	
	Brukarar av MacOS, eller Linux kan også nytta MikTeX, men har også andre alternativ. For MacOS: \url{https://www.tug.org/mactex/}
	
	For Linux vil eg heller anbefala at ein installerer \textit{TeX Live}. Bruk google, eller spør om hjelp om du ikkje finn ut av det sjølv.
\end{frame}


%%% Local Variables:
%%% mode: latex
%%% TeX-master: "../latex-presentation"
%%% End:

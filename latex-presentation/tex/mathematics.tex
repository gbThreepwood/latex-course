\section{Matematikk}

	\begin{frame}[containsverbatim]{Matematikk}
	
	Dersom ein jobbar med ingeniørfag vil det ofte vera behov for å beskriva matematiske uttrykk. Det er derfor viktig å ha eit fleksibelt og enkelt system for dette. Dette er ein av dei største styrkene til \LaTeX, og det er faktisk mulig å nytta den samme syntaksen for å setta inn formlar i Microsoft Word. For å typesetta fylgjande uttrykk:
	
	\begin{equation}
		\int_0^\infty \mathrm{e}^{-x}\,\mathrm{d}x
	\end{equation}
	
	Kan ein nytta fylgjande kode:
	
	\begin{minted}{latex}
\begin{equation}
	\int_0^\infty \mathrm{e}^{-x}\,\mathrm{d}x
\end{equation}
	\end{minted}
	
\end{frame}




\begin{frame}[containsverbatim]{Matematikk}
	
	Dersom ein ikkje ynskjer nummerering kan ein skriva:
	
	\begin{minted}{latex}
\begin{equation*}
	\int_0^\infty \mathrm{e}^{-x}\,\mathrm{d}x
\end{equation*}
	\end{minted}
	
	For å plassera matematikk inne i teksten kan ein nytta:
	
	\mintinline{latex}|\(a = \frac{b}{c}\)|, som gir: \(a = \frac{b}{c}\)
	
	Unngå bruk av dollarteikn (\$) sjølv om det også fungerer:
	\mintinline{latex}|$a = \frac{b}{c}$|, som gir: $a = \frac{b}{c}$
	
	Dollarteikn er ein \TeX{} kommando, og vil ikkje gi like gode feilmeldingar som \mintinline{latex}|\(\)| i \LaTeX{}.
	
	
	
\end{frame}



\begin{frame}[containsverbatim]{Avansert matematikk}
	
	\begin{equation}
		z = \overbrace{
			\underbrace{x}_\text{reell} + j
			\underbrace{y}_\text{lateral}
		}^\text{komplekst tal}
	\end{equation}
	
	\begin{minted}{latex}
\begin{equation}
	z = \overbrace{
		\underbrace{x}_\text{reell} + j
		\underbrace{y}_\text{lateral}
	}^\text{komplekst tal}
\end{equation}
	\end{minted}
	
\end{frame}